\PassOptionsToPackage{utf8}{inputenc}
\documentclass{bioinfo}
\copyrightyear{2018} \pubyear{2018}

\access{Advance Access Publication Date: Day Month Year}
\appnotes{Applications Note}

\begin{document}
\firstpage{1}

\subtitle{Databases and ontologies}

\title[Bio2BEL]{Bio2BEL: Integration of Structured Knowledge Sources with Biological Expression Language}
\author[Hoyt \textit{et~al}.]{Charles Tapley Hoyt\,$^{\text{\sfb 1,2,*}}$, Christian Ebeling\,$^{\text{\sfb 1}}$, Daniel Domingo-Fern{\'{a}}ndez\,$^{\text{\sfb 1,2}}$, Andrej Konotopez\,$^{\text{\sfb 1}}$, Mohammed Asif Emon\,$^{\text{\sfb 1,2}}$, Colin Birkinbihl\,$^{\text{\sfb 1,2}}$, {\"{O}}zlem Muslu\,$^{\text{\sfb 1,2}}$, and Martin Hofmann-Apitius\,$^{\text{\sfb 1,2}}$}
\address{$^{\text{\sf 1}}$Department of Bioinformatics, Fraunhofer Institute for Algorithms and Scientific Computing (SCAI), Sankt Augustin, 53754, Germany \\
$^{\text{\sf 2}}$Bonn-Aachen International Center for IT, Rheinische Friedrich-Wilhelms-Universit{\"a}t Bonn, Bonn, 53113, Germany.}

\corresp{$^\ast$To whom correspondence should be addressed.}

\history{Received on XXXXX; revised on XXXXX; accepted on XXXXX}

\editor{Associate Editor: XXXXXXX}

\abstract{\textbf{Motivation:} It is now necessary to integrate heterogeneous, multiscale, and multimodal data to support joint analysis and to unravel the mechanisms and aetiology of complex disease. Because of its unique ability to capture this variety, Biological Expression Language is well suited to be further used as a platform for semantic integration and harmonization in networks and systems biology. \\
\textbf{Results:} We have developed several independent packages capable of downloading, structuring, and serializing various biological resources to Biological Expression Language.\\
\textbf{Availability:}  Each Bio2BEL repository is implemented Python code and distributed through https://github.com/bio2bel.\\
\textbf{Contact:} \href{charles.hoyt@scai.fraunhofer.de}{charles.hoyt@scai.fraunhofer.de}\\
\textbf{Supplementary information:} Supplementary data are available at \textit{Bioinformatics} online.}

\maketitle

\section{Introduction}

The integration of heterogeneous, multi-scale, and multi-modal biomedical data is a modern necessity in the quest to unravel the mechanisms and aetiology of complex disease. 
An overarching strategy was proposed as early as 1995 that outlined the transformation data into a common model, semantic alignment of related objects, integration of schemata, and federation of data \citep{Davidson1995}.
However, integration remains both a complicated and complex task that requires the identification of resources and their formats, conversion, harmonization, and unification.

Interest in the semantic web and linked open data along with the adoption of RDF in the biomedical community led to the Bio2RDF project, in which pipelines for serializing several biological data- and knowledge bases to RDF were developed \citep{Belleau2008}. However, a one-size-fits-all solution for integration in both a semantic and biologically meaningful way is nothing short of a pipe dream. 

As stated before, the expressive power of RDF is counteracted by its lack of domain 
specificity. While it can be used as an interchange format, it still requires converters 
to formats for which analytical pipelines have already been developed. Furthermore, the 
suite of conversion scripts are written in PHP, which has very little traction in the 
bioinformatics community and therefore is difficult to integrate in pre-existing workflows. 


The foray into new disease areas and clinical indications has necessitated the assembly of knowledge on wider scales from the genetic to the phenotypic and population levels. While most modeling languages and data formats for assembling knowledge are insufficient for such a task, BEL possesses the faculty for capturing multi-scale knowledge.

In the same way Pathway Commons was successful at combining many molecular pathway and interaction database, BEL has the potential to serve as a semantic integration platform through which knowledge and data across scales can be integrated and analyzed. BEL has already been used to reason over the previously untapped sources of chemogenomic and chemical genetic information  \citep{Emon2017} and can be extended further into the realm of disease-disease, disease-protein, disease-chemical, and chemical-chemical networks.

Enabled by PyBEL \citep{Hoyt2017}

Several software packages have been released (

Orange Bioinformatics \citep{Demsar2013}
BioServices \citep{Cokelaer2013}
BioPython \citep{Cock2009}

\section{Approach}

Each Bio2BEL repository has five components: 1) a downloader, 2) a parser, 3) a database model, and 4) tools for exporting to BEL.

Advent of PyBEL library makes everything amazing

Among the most easily integrable structured knowledge formats in BEL are taxonomies, ontologies, and networks. Taxonomies and ontologies directly provide the facility for reasoning and inferences of new knowledge. Networks, such as bipartite SNP-disease, chemical-gene, or gene-pathway networks, can be directly integrated in BEL. Even networks created by statistical calculations can be added to BEL networks to investigate their explanatory power. For example, the eQTL Single Nucleotide Polymorphism Ontology (eSNPO) provides statistical associations between SNPs and Gene Ontology (GO) biological processes.

\begin{methods}
\section{Methods}

Each repository is packaged so that it can be distributed through PyPI. They share common programmatic interfaces and each implement a Command Line Interface. They are all documented in the same style with ReadTheDocs.

Common resources for serializing BEL namespaces and annotations are stored within the pybel.resources module. Additionally, since its initial publication, we have implemented an internal domain-specific language in the PyBEL package to support easily readable generation of BEL statements in Python programs.

\begin{table}[!t]
\processtable{Select Repositories Implemented in Bio2BEL\label{Tab:01}} {\begin{tabular}{@{}lll@{}}\toprule
Name & Format & Knowledge\\\midrule
HGNC & JSON & Gene orthologies, equivalencies, and families\\
FlyBase & Tabular & Gene orthologies and equivalences\\
RGD & Tabular & Gene orthologies and equivalences\\
MGI & Tabular & Gene orthologies and equivalences\\
Entrez/Homologene & Table & Gene/protein orthologies and equivalences\\
UniProt & XML & \\
ExPASy & Custom & Enzyme class hierarchy and protein membership\\
InterPro & Custom & Protein families, domains, and ProSites\\
ChEBI & Tabular & Chemical equivalences and hierarchies\\
ChEMBL & SQL & Biochemical activity\\
CTD & XML & Chemical activity\\
BKMS & Custom & Chemical participation in reactions\\
HMDB & Custom & \\
miRBase & Tabular & Gene orthologies and equivalences\\
miRTarBase & Tabular & miRNA-Target interactions\\
GO & OBO & Biological process hierarchies\\
HP & OLS & Phenotype hierarchy\\
UBERON & OLS & Anatomical term hierarchy\\
MeSH & Custom & Multi-type hierarchies\\\botrule
\end{tabular}}{This is a footnote}
\end{table}

\end{methods}

\section{Case Study}

Let's take ChEBI as a case study. Can show nice pictures from the Statin graph example
that show that a Bio2BEL package enables complex logic to be written to reason over
the specific contents of a database, and enable cool inference (can be shown in two 
side-by-side pictures)

\section{Discussion}

One of the advantages of BEL over other systems biology modeling languages is its ability to model knowledge across modes and scales. As it is used to describe new phenotypes, such as the domain of psychiatry, new namespaces must be identified and formatted. Below, two approaches for building new namespaces are described.

\section*{Acknowledgements}

...

\section*{Funding}

This work was supported by the EU/EFPIA Innovative Medicines Initiative Joint Undertaking under AETIONOMY [grant number 115568], resources of which are composed of financial contribution from the European Union's Seventh Framework Programme (FP7/2007-2013) and EFPIA companies in kind contribution.

Conflict of Interest: none declared.

\bibliographystyle{natbib}
\bibliography{document}

\end{document}
